\begin{frame}{Базовая работа с Git}{Добавление файлов в индекс}
    \begin{itemize}
        \item
              Для добавления файла в индекс используется команда git add
        \item
              Файлы можно добавлять выборочно, тогда необходимо указать пути к файлам с командой git add
        \item
              Можно также добавить все текущие изменения: для этого используется запись git add . или git add -u
        \item
              После изменения можно проверить состояние командой git status -- она должна показать все изменения, добавленные в индекс. Разницу между индексом и последним коммитом можно получить, выполнив git diff --staged
        \item
              Если файл, изменения которого уже в индексе, опять изменить, новые изменения надо добавлять отдельно
    \end{itemize}
\end{frame}