\begin{frame}{Базовая работа с Git}{Добавление файлов в индекс}
    \begin{itemize}
        \item
              Для добавления файла в индекс используется команда \lstinline[style=BashInputStyle]{git add}
        \item
              Файлы можно добавлять выборочно, тогда необходимо указать пути к файлам с командой \lstinline[style=BashInputStyle]{git add}
        \item
              Можно также добавить все текущие изменения: для этого используется запись \lstinline[style=BashInputStyle]{git add .} или \lstinline[style=BashInputStyle]{git add -u}
        \item
              После изменения можно проверить состояние командой \lstinline[style=BashInputStyle]{git status} -- она должна показать все изменения, добавленные в индекс. Разницу между индексом и последним коммитом можно получить, выполнив \lstinline[style=BashInputStyle]{git diff --staged}
        \item
              Если файл, изменения которого уже в индексе, опять изменить, новые изменения надо добавлять отдельно
    \end{itemize}
\end{frame}