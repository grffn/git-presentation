\begin{frame}{Структура Git}
    Репозиторий Git можно разделить на три зоны:
    \begin{itemize}
        \item
              Git-директория (.git) — это то место, где Git хранит метаданные и базу объектов вашего проекта (папка .git)/
        \item
              Рабочая директория (Working directory) -- это директория, в которой располгаются все файлы текущей версии проекта. Тут происходит вся основная работа
        \item
              Область подготовленных файлов (Staging area) — это набор всех изменений, которые попадут в новый коммит. Эту область ещё называют “индекс”.
    \end{itemize}
\end{frame}