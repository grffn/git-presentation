\begin{frame}{Базовая работа с Git}{Создание коммита}
    \begin{itemize}
        \item
              Коммит создается командой git commit. В коммит добавляются все изменения, находящиеся в индексе
        \item
              Коммит невозможно (почти) создать, не указав сообщение -- краткое его описание. После вызова команды git commit откроется окно редактора, в котором нужно будет указать сообщение
        \item
              Сообщение можно задать при вызове команды git commit, указав флаг -m, например git commit -m "Message"
        \item
              Указав флаг -a можно произвести добавление всех изменений, кроме добавления новых файлов, в индекс и затем произвести коммит
    \end{itemize}
\end{frame}