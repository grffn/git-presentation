\begin{frame}{Базовая работа с Git}{Журнал коммитов}
    \begin{columns}
        \begin{column}{0.6\textwidth}
            \begin{itemize}
                \item
                      Все созданные коммиты помжно просмотреть в журнале коммитов (git log)
                \item
                      По умолчанию, коммиты будут выведены в обратном хронологическом порядке, начиная с последнего. Выводятся SHA-1 контрольная сумма коммита, время создания, а также данные автора
                \item
                      Параметры вывода можно изменять при помощи задания флагов при вызове git log
            \end{itemize}
        \end{column}
        \begin{column}{0.4\textwidth}
            \begin{figure}
                \centering
                \includegraphics[width=\textwidth]{images/git-log-example.png}
                \caption{Пример git log}
            \end{figure}
        \end{column}
    \end{columns}

\end{frame}